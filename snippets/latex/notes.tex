\documentclass{article}
%\documentclass[a4paper, 12pt]{article}
%\documentclass[a4paper, 12pt, draft]{article} % don't include images, leave a border of same size...great

% Variable de compilation
\newif\ifbeamer
\beamerfalse
\newcommand{\beamer}[2]{\ifbeamer #1 \else #2 \fi}



%\usepackage[latin1]{inputenc}
%\usepackage[utf8]{inputenc} % manage utf8 encodage 
%\usepackage[french]{babel} % for french document ! dirty enumerate style,+ bad change rectangle colors for section linking.


%\usepackage[fleqn]{amsmath}
\usepackage{amsmath,amssymb,amsbsy,amsfonts}
\usepackage[ruled,vlined]{algorithm2e} % what for .
\usepackage{fancyhdr} % for heading
\usepackage[colorlinks=true, urlcolor=blue]{hyperref} % url, link
\usepackage{geometry}
\usepackage{listings}
\usepackage{enumerate}
\usepackage{url}
\usepackage{booktabs}
\usepackage{multirow}
\usepackage{multicol}
\usepackage{fancyvrb}
\usepackage{yfonts}
\usepackage{adjustbox}
\usepackage{graphicx}
\usepackage{subfigure}
\usepackage{wrapfig}
\usepackage{calc}%    For the \widthof macro
\usepackage{xparse}%  For \NewDocumentCommand
\newcommand{\tikzmark}[1]{\tikz[overlay,remember picture] \node (#1) {};} % what for ?
\usepackage{tikz}
\usetikzlibrary{bayesnet}
%%%%%%%%%%%%%%%%%%%%%%%%%%%%%%%%%%%%%%%%%%%%%%%%%%%%%%%%
%%%%% => Flow Graph Environnement (TIKZ)
%%%%%%%%%%%%%%%%%%%%%%%%%%%%%%%%%%%%%%%%%%%%%%%%%%%%%%%

\usetikzlibrary{trees}
%\usetikzlibrary{graphs}
\usetikzlibrary{automata}
\usetikzlibrary{positioning}
\usetikzlibrary{petri}
\usetikzlibrary{arrows}
\usetikzlibrary{calc}
\usetikzlibrary{decorations.pathreplacing}
\usetikzlibrary{decorations.markings}
\usetikzlibrary{backgrounds}
\usetikzlibrary{fit}
\usetikzlibrary{snakes}
\usetikzlibrary{matrix}
\usetikzlibrary{shapes.geometric}
\usetikzlibrary{shapes.symbols}

%%%%%%%%%%%%%%%%%%%%%%%%%%%%%%%%%%%%%%%%%%%%%%%%%%%%%%%
%%%%% => Initialisation.
%%%%%%%%%%%%%%%%%%%%%%%%%%%%%%%%%%%%%%%%%%%%%%%%%%%%%%%

% Constant, dimensions.
\newcommand{\scale}{1}
\newcommand{\mfont}{\large}
\newlength{\unit} \setlength{\unit}{\scale cm} 
\newlength{\lbloc} \setlength{\lbloc}{2cm} \newlength{\hbloc} \setlength{\hbloc}{1cm} % blocs size
\newlength{\ebloc} \setlength{\ebloc}{3cm} % space between blocs
\newlength{\ltext} \setlength{\ltext}{2 cm} % longeur de text max (breakline)
\beamer{\newcommand{\ymove}{2.5*\unit}} {\newcommand{\ymove}{2.5cm}}

% Text
\tikzstyle{textc}=  [text centered, text width=\ltext, font=\mfont] % \tikzstyle{text}=[align=center] % for "up to date"version
\tikzstyle{textj}=[text justified,text width=\ltext, font=\mfont] 
\tikzstyle{textit}=[font=\itshape\small] 

% default style
\tikzstyle{css}=[rounded corners=3pt, textc, minimum size=\lbloc, minimum height=\hbloc] %node distance=\ebloc/2,

% Bloc
\tikzstyle{bloc}=[draw=black,rectangle,rounded corners=3pt]
\tikzstyle{wbloc}=[draw=white,rectangle,rounded corners=3pt]
\tikzstyle{sbloc}=[draw,rectangle,rounded corners=3pt, textc ]
% Carre
\tikzstyle{carre}=[draw=black, rectangle, css]
% Losange
\tikzstyle{losange}=[draw=black, diamond, css]
% Rond
\tikzstyle{rond}=[draw=black, ellipse, css]	
% Cloud
\tikzstyle{cloud}=[rond, dashed]	
% Double Carre
\tikzstyle{ccarre}=[bloct, thick, double] 

% CSS
\newlength{\dline} \setlength{\dline}{2pt} % double line distance

% Line
\tikzstyle{line}=[-, thick]
\tikzstyle{arrow}=[->, thick]
\tikzstyle{rarrow}=[<-, thick]
\tikzstyle{waves}=[decorate, decoration={snake, amplitude=.5mm,segment length=2mm,post length=3pt}]
\tikzstyle{brace}=[decorate, decoration={brace, mirror}]
\tikzstyle{flux}=[->, double distance=\dline]
\tikzstyle{rflux}=[<-, double distance=\dline]
\tikzstyle{dflux}=[<->, double distance=\dline]
\tikzstyle{ffflux} = [thick, decoration={markings,mark=at position
   1 with {\arrow[semithick]{open triangle 60}}},
   double distance=1.4pt, shorten >= 5.5pt,
   preaction = {decorate},
   postaction = {draw,line width=1.4pt, white,shorten >= 4.5pt}]
\tikzstyle{innerWhite} = [semithick, white,line width=1.4pt, shorten <= 4.5pt]

% build a antenna
\newcommand{\antensize}{0.4}
\newcommand{\antena}[1]
{\draw  (#1) -- ++(50:\antensize)
        (#1) -- ++(90:\antensize)
        (#1) -- ++(130:\antensize);
}



\geometry{a4paper,
    body={160mm,260mm},
    left=25mm,top=20mm,
    headheight=4mm,headsep=8mm,
    footskip=10mm,
}
                                              

\newtheorem{definition}{Definition}[section]
\newtheorem{proposition}{Proposition}[section]
\newtheorem{theorem}{Theorem}[section]
\newtheorem{corollary}{Corollary}[section]
\newtheorem{proof}{Proof}[section]

\renewcommand{\labelitemi}{$\bullet$}
\renewcommand{\labelitemii}{$\cdot$}
\renewcommand{\labelitemiii}{$\diamond$}
\renewcommand{\labelitemiv}{$\ast$}

% equation reference
\renewcommand{\theequation}{\thesection.\arabic{equation}}


% write code
\lstnewenvironment{C}[1]
{\lstset{language=C,
      frame=tBRl,
      basicstyle=\scriptsize,stringstyle=\emph,showstringspaces=false,
      numbers=left,numberstyle=\tiny,
      breaklines=true, columns=flexible, title={#1}}
}{}

% Preambles Pages
\pagestyle{fancy}
\fancyhf{} % remove default headers
\fancyfoot[R]{\thepage}
\renewcommand{\footrulewidth}{0.3pt}
\renewcommand{\headrulewidth}{0.3pt}


\title{ %\vspace{2cm}
      \Large{Infinite Latent Features Relational Model}\\--\\ }

\author{Smith}
\date{avril 2015}


\begin{document}
\fancypagestyle{plain}{
      \fancyhf{}
      %\fancyhead[L]{\includegraphics[scale=0.4]{ipb.eps}}
      \renewcommand{\footrulewidth}{0pt}
      \renewcommand{\headrulewidth}{0pt}
}
\maketitle


\section{Introduction}

Let $\mathcal{Y}$ be the set of all relational observations or links, we assume that the links are drawn from independant Bernoulli densities given $\pi_{ij}$:



\bibliographystyle{unsrt}
\bibliography{a}

\end{document}











%%%%%%%%%%%%%%%%%%%%%%%%%%%%%%%%%%%%%%%%%
%%%%%%%%%%%%%%%%%%%%%%%%%%%%%%%%%%%%%%%%%
%%%%%%%%%%%%%%%%%%%%%%%%%%%%%%%%%%%%%%%%%
%%%%%%%%%%%%%%%%%%%%%%%%%%%%%%%%%%%%%%%%%
%%%%%%%%%%                   %%%%%%%%%%%%
%%%%%%%%%%                   %%%%%%%%%%%%
%%%%%%%%%%                   %%%%%%%%%%%%
%%%%%%%%%%                   %%%%%%%%%%%%
%%%%%%%%%%      examples     %%%%%%%%%%%%
%%%%%%%%%%                   %%%%%%%%%%%%
%%%%%%%%%%                   %%%%%%%%%%%%
%%%%%%%%%%                   %%%%%%%%%%%%
%%%%%%%%%%                   %%%%%%%%%%%%
%%%%%%%%%%                   %%%%%%%%%%%%
%%%%%%%%%%%%%%%%%%%%%%%%%%%%%%%%%%%%%%%%%
%%%%%%%%%%%%%%%%%%%%%%%%%%%%%%%%%%%%%%%%%
%%%%%%%%%%%%%%%%%%%%%%%%%%%%%%%%%%%%%%%%%
%%%%%%%%%%%%%%%%%%%%%%%%%%%%%%%%%%%%%%%%%

\begin{Verbatim}[fontsize=\small]
(:action move
  :parameters (?thing ?from ?to)
  :precondition (and (road ?from ?to)
     (at ?thing ?from) (mobile ?thing)
     (not (= ?from ?to)))
  :effect (and (at ?thing ?to)
     (not (at ?thing ?from))))
\end{Verbatim}
\begin{algorithm}

%\dontprintsemicolon
\KwIn{Two actions $a_1$ and $a_2$ to merge}
\KwOut{The macro-action $m$}
\Begin{
$m \gets a_1$\;
\ForEach{precondition $p \in \mathrm{Pre}(a_2)$}{
\If{$ p \notin \mathrm{Add}(m) \cup \mathrm{Pre}(m)$}{
	$\mathrm{Pre}(m) \gets \mathrm{Pre}(m) \cup \{p\}$\;
	}
}
\ForEach{delete effects $d \in \mathrm{Del}(a_2)$}{
\eIf{$ d \in \mathrm{Add}(m)$}{
	$\mathrm{Add}(m) \gets \mathrm{Add}(m) \backslash \{d\}$\;
	}
	{$\mathrm{Del}(m) \gets \mathrm{Del}(m) \cup \{d\}$\;}
}
\Return{$m$}\;}
\caption{merge$(a_1, a_2)$}
\label{algo:merge}
\end{algorithm}

% Vertical Split with line
\begin{minipage}[t]{0.35\linewidth}
    \flushleft{
        \begin{align*}
           Y | \Pi &\sim Bernoulli(\pi_{ij}) \\
            F | \alpha &\sim \mathrm{IPB}(\alpha) \\
            w_{kl} | \sigma_w &\sim \mathcal{N}(0, \sigma_w^2) 
        \end{align*}
}
\end{minipage}
\qquad{\color{black}\vrule}\qquad
\begin{minipage}[t]{0.45\linewidth}
    \begin{align*}
    P(Y | \Pi ) = \prod_{(i,j)\in \mathcal{Y}} \pi_{ij}^{y_{ij}} (1-\pi_{ij})^{1-y_{ij}} \\
    \pi_{ij} = p(y_{ij} = 1 | F, W ) = \sigma(F_i^\top WF_j)
    \end{align*}
\end{minipage}

% equation and box
\[bits \longrightarrow \fbox{emetteur} \xrightarrow{e(t)} \parbox[b]{15ex}{$propagation$ \newline \framebox[15ex]{$H(t)$}} \longrightarrow \parbox[t]{4mm}{$\oplus$\\$\uparrow$\\$n(t)$} \xrightarrow{s(t)} \fbox{recepteur} \longrightarrow \hat{bits}\]
$e(t)$: signal emis.\\
$s(t)$: signal re�us.\\
$n(t)$: bruit banc gaussien.\\


% figure flottante, title, label (ref)
\begin{figure}[h]
\centering
\includegraphics[scale=0.4]{tv-map.eps}
\caption{\textbf{Standard de broadcast video}}
\label{fig:tv-map}
\end{figure}

% figure a cot� du texte
\begin{wrapfigure}{r}{0.35\linewidth}
\includegraphics[scale=0.3]{couloir.eps}
\end{wrapfigure}

% two figure by side
\begin{figure}[h]
\begin{minipage}{0.45\linewidth}
\includegraphics[scale=0.39]{noise.eps}
\end{minipage}
\begin{minipage}{0.45\linewidth}
\includegraphics[scale=0.39]{noise_p.eps}
\end{minipage}
\caption{un echantillons de bruit et sa densit� de probabilit� gaussienne}
\end{figure}

% tableau resised (box)
% with title and flottant ([h)
\begin{table}[h]
\caption{Caract�ristique de diff�rentes version de l'USRP}
\resizebox{\textwidth}{!}{
\begin{tabular}{|c|c|c|c|c|c|c|c|c|}
\hline
Model& RF channels&  Host Intf& Host BW(MHz)&    DAC&                 ADC&              MIMO&  CPU&       Cost\\ 
\hline
N200&  1TX/1RX&      GigE&           50&      16-bit, 400 MSPS&    14-bit, 100 MSPS&    Yes**& n/a&       \$1,500\\
N210&  1TX/1RX&      GigE&           50&      16-bit, 400 MSPS&    14-bit, 100 MSPS&    Yes**& n/a&       \$1,700\\
E100*& 1TX/1RX&      Embedded&       4-8&     14-bit, 128 MSPS&    12-bit, 64 MSPS &    No&    OMAP 3730& \$1,300\\
E110*& 1TX/1RX&      Embedded&       4-8&     14-bit, 128 MSPS&    12-bit, 64 MSPS &    No&    OMAP 3730& \$1,500\\
USRP1& 2TX/2RX&      USB 2.0&         16&     14-bit, 128 MSPS&   12-bit, 64 MSPS  &    Yes&   n/a&       \$700\\
B100&  1TX/1RX&      USB 2.0&         16&     14-bit, 128 MSPS&   12-bit, 64 MSPS  &   No&    n/a&       \$650\\
\hline
\end{tabular}}
\end{table}

% note de marge
\marginpar{test \\ test2}

% itemize option
\renewcommand{\labelitemi}{$\bullet$}
\renewcommand{\labelitemii}{$\cdot$}
\renewcommand{\labelitemiii}{$\diamond$}
\renewcommand{\labelitemiv}{$\ast$}
